% In this file you should put the actual content of the blueprint.
% It will be used both by the web and the print version.
% It should *not* include the \begin{document}
%
% If you want to split the blueprint content into several files then
% the current file can be a simple sequence of \input. Otherwise It
% can start with a \section or \chapter for instance.


\chapter{Big Picture}


In (the first big part of) this project, we will formalize the topos semantics of geometric logic, where `topos' refer to a Grothendieck topos.

We formalize sequentially
\begin{itemize}
  \item a deep embedding of (multi-sorted) syntax of geometric logic
  \item interpretation of geometric syntax on a geometric category
  \item soundness of geometric semantics
  \item soundness of topos semantics
  \item completeness of topos semantics
\end{itemize}

As we have both soundness and completeness, we have a central milestone of this project:
A geometric sequent is constructively provable if and only if it holds in any Grothendieck topos.

The `only if' direction is for the positive use of the semantics---it gives us a shortcut to prove non-trivial result in a geometric category with complicated construction.
The `if' diection is for the negative use of the semantics---to show a sequent is not provable, it is sufficient to give an example of a topos where it does not hold.

We will give examples for both ways of the application.



Here is a list of choices that we made:
\begin{itemize}
  \item We use de Brujin indices instead of name-carrying for variables in our syntax
  \item We close the sorts under finite product
  \item We use sequent calculi instead of Hilbert-style proof rules
  \item We allow infinite many sorts, which is reasonable for the application on internal logic of any category
  \item We employ the construction of syntactic site in Wrigley's PhD work, which is equivalent to the standard construction
  in Johnstone
  \item We use an intrisic characterization of sheaf topos in terms of axioms. In particular, the objects in
  our topos are not a priori functors

\end{itemize}


Our plan is to split the project into three big parts:
\begin{itemize}
  \item soundness and completeness proof among geometric logic
  \item completeness among Grothendieck topoi via syntactic site
  \item applications of soundness and completeness
\end{itemize}

\chapter{Monosorted Geometric Logic}
We aim for multi-sorted geometric logic eventually. But the part of completeness proof of our project is rather complicated, we stick to
monosorted signature in our experimental stage. We will integrate it into multisorted case once the monosorted case is done.

\section{Geometric Signature, Theories and Logic}

\begin{definition}
  %Lean code should be a datatype, how to link that?
A geometric signature contains the information of
\end{definition}

%\section{Syntactic Site for a Geometric Theory} Previously @KM: is that okay?


For the whole section, we fix a geometric theory \thT{} over some universe \Univ{}.

\section{Syntactic Category}
\begin{definition}[Syntactic Category]
  %\lean{Joshua.fmlInCtx, Joshua.fmlMap, Joshua.categoryStruct}
  \leanok
  The syntactic category associated to $\mathcal{T}$ has:
  \begin{itemize}
    \item An object is of the form $\xphi$ where $\vx$ is a context consists of variables $x_1, \ldots, x_{n}$ and $\phi $ is a geometric
    formula under this context. i.e. $\phi$ contains only free variables from $\vx$.
    %as objects, pairs $\fmlInCtx{\vec{x}}{\varphi}$ of a context $\vec{x} \coloneq x_1, \ldots, x_{n}$ of variables and a geometric formula $\vec{x} \vdash_{\thT} \varphi$ over \thT{};
    \item A map $\xphi\to \ypsi$ is a renaming $\vy\to \vx$ satisfying some extra condition.
    A \emph{renaming} $\vy\to \vx$ is a function $f:\{y_1,\ldots,y_m\}\to \{x_1,\ldots,x_n\}$ that maps variables to variables.
    Such a renaming defines a map $\xphi\to \ypsi$ if $\phi\vdash_{\vx}\psi[x_{f(i)}/y_i]$, means that as formulas in the context $\vx$, we have $\phi$
    proves $\psi[x_{f(i)}/y_i]$ in the theory  \thT{}.
    %as morphisms between $\fmlInCtx{\vec{x}}{\varphi}$ and $\fmlInCtx{\vec{y}}{\psi}$, a renaming $\rho : \vec{y} \to \vec{x}$ such that $\varphi \vdash_{\vx} {\psi}[x_{\rho(i)}/y_i]$.
  \end{itemize}
\end{definition}



\section{Syntactic Site}
\begin{definition}[Covering family]
  \label{Def:CoveringFamily}
  \lean{WrigleyTopology.CoveringFamily}
  \leanok
  A family of maps $(\sigma_k : \fmlInCtx{\vec{x}_k}{\varphi_k}\to \fmlInCtx{\vec{y}}{\psi})_{k \in K}$ with $K \in \Univ$ is covering when
  $\psi \vdash_{\vy} \bigvee_{k \in K} (\exists \vec{x}_k, \phi \wedge \bigwedge_{i \in \vec{y}} y_i = x_{\sigma_{k(i)}}$).
\end{definition}

This covering family defines a syntactic site:

\begin{definition}[Syntactic Site]
  \lean{WrigleyTopology.SyntacticTopology}
  A sieve is a covering sieve when it contains a covering family as defined in Definition \ref{Def:CoveringFamily}.
\end{definition}

To prove Definition \ref{Def:CoveringFamily} gives a topology, we need to check the three conditions as in the definition of Grothendieck
topology.
We prove these three conditions in the following three subsections.

\subsection{Identity}
\begin{lemma}[Identity map its own consists of a covering family]
  \lean{WrigleyTopology.id_covers}
  For each object $\xphi$ in the syntactic category, the identity map $\{\mathbf{1}_{\xphi}\}$ is a covering family.
\end{lemma}

\subsection{Transitivity}
\begin{lemma}[Transitivity]
  \lean{WrigleyTopology.Transitivity.transitivity}
  Given a formula in context $\xphi$ and a two sieves $S,R$ on it, where $S$ is a covering sieve, then if for each arrow $f:\ypsi\to\xphi$ in the
  covering sieve $S$, the pullback of $R$ along $f$ is a covering sieve on $\ypsi$, then $R$ itself is a covering sieve.
\end{lemma}

\begin{proof}
  As $S$ is a covering sieve, $S$ contains a covering family, say the family $\{\sigma^i: \ypsi^i\to \xphi\}$ indexed by $i\in I$.
  By definition of covering family, we have
  \begin{equation}\label{seq1}
    \phi \vdash_{\vx} \bigvee^{i\in I} (\exists \vy^i. \;\psi^i \wedge \bigwedge_{n \in \vx} x_n = {y^i}_{\sigma^i(n)})
  \end{equation}

  The assumption says for each $\sigma^i: \{\vy^i\mid \psi^i\} \to \xphi$, the pullback' of $R$ along $\sigma^i$ is a covering sieve
  on  $\{\vy^i\mid \psi^i\}$ .
  This means each such sieve contains a covering family. Say, for each $i\in I$, we have a set $J^i$ indexing this covering family, and
  the family is written as $\{\rho_j: \zzeta_j\to \ypsi^i\}$ then we have

  \begin{equation}\label{seq2}
    \psi^i \vdash_{\vy^i} \bigvee_{j\in J^i}
     (\exists \vz_j.\; \zeta_j \wedge \bigwedge_{m \in \vy^i} y_m = {z_j}_{\rho_j(m)})
  \end{equation}

  Then we obtain a family indexing by $\Sigma_{i\in I}J^i$ that covers $\xphi$. The arrows are of the form:
  \begin{center}
  \begin{tikzcd}
    \zzeta^i_j\ar[r,"\rho^i_j"] & \ypsi^i\ar[r,"\sigma^i"] & \xphi
  \end{tikzcd}
  \end{center}
  Recall that $\rho^i_j$ is in the pullback' of the sieve $R$ along $\sigma^i$, so we have the composition $\sigma^i\circ \rho^i_j\in R$.
  Such a family covers $\xphi$. It requires us to prove

  \begin{equation*}
    \phi \vdash_{\vx} \bigvee^{(i,j)\in \Sigma_{i\in I}J^i}
    (\exists \vz^i_j.\; \zeta^i_j \wedge \bigwedge_{n\in \vx} x_n = {z^i_j}_{\rho^i_j\circ \sigma^i(n)})
  \end{equation*}

  the proof is straightforward by combining \ref{seq1} and \ref{seq2} above.



\end{proof}

For this proof, the trickest part is the construction of the index of the covering family in $R$, i.e. the dependent sum $\Sigma_{i\in I}J^i$.
It need to be a term of our small universe $\Univ$. We equip our small universe with the following two ingredients, which will be enough for this proof.

\begin{itemize}
  \item $\Univ$ is closed under dependent sum. We axiomatize a `small object former' that makes $\Sigma$-objects in $U$ by
  taking a small object $A\in \Univ$ and a family $B:A\to Type$ indexed by $A$. Accordingly, we add axioms formulating
  the projections and the computation rules of the dependent sum in $\Univ$.
  \item $\Univ$ admits a form of axiom of choice. More formally, for each term $A\in \Univ$ representing a
  small object, an a small family of types indexed by $A$, if each of $X_a$ for $a\in A$ is nonempty, then the type
  $\Pi_{a\in A}X_a$ is nonempty.
\end{itemize}


The reason that we need the axiom of choice is that our constructor for the $\Sigma$-object in $\Univ$ asks
for a term $B: El(A)\to \Univ$. However, the assumption only gives that for each index in $A$, there exists a $\Univ$-small
type indexing a family, it does not directly give a function term $El(A)\to \Univ$.





\subsection{Stability}

The definition of stability is as follows: For a site $(C,J)$, the covering $J$ is stable under pullback if for a arrow $f:x\to y$ in $C$,
if $S\in J(y)$, i.e. $S$, is a covering sieve on $y$, then the pullback $f^*(S)$ along $f$ is a covering sieve on $x$, i.e. $f^*(S) \in J(x)$.

Note that the notion of ``pullback of sieves'' is not in terms of pullback squares by definition. The definition is:


\begin{equation*}
  f^*(S) := \{a: t \to x \mid (f\circ a: t\to y) \in S\}
\end{equation*}

To be super clear, whenever we say refer to ``pullback of sieves'' under this definition in English words, rather than ``pullback squares'', we will write it
as pullback'.
In our case, we want to prove that if we have a covering sieve $\{g_K\}\to \ypsi$, then its pullback' is a covering sieve.
That is, if we have a covering family in $\{g_K\}$, covering $\ypsi$ as in definition \ref{Def:CoveringFamily}, then the pullback'
$f^*\{g_K\}$ contains a covering family.

We use the fact that if $\{g_k\}\subseteq \{g_K\}$ is the covering family, then the covering family contained in $f^*\{g_K\}$ has arrows obtained in the following
way: for each $k$, consider the pullback square:
\begin{center}
  \begin{tikzcd}
     f^* \zzeta \ar[r,"\iota_2^k"]\ar[d,"f^*(g_k):= \iota_1^k"]  & \zzeta\ar[d,"g_k"]\\
     \xphi \ar[r,"f"] & \ypsi
  \end{tikzcd}
\end{center}

then the covering family consists of arrows of the form $\iota_1^k$ as in the diagram above.

In fact, we are not using the universal property of pullbacks in our proof. Instead, we effectively construct a pullback structure and use
one of the projection maps as the covers we want, without proving that what we construct is actually a pullback.

The construction of pullbacks in the syntactic category is as follows. For a pair of arrows $f: \xphi\to \ypsi$ and $g:\zzeta\to \ypsi$,
the pullback of $g$ along $f$ is pictured as:


\begin{center}
  \begin{tikzcd}
     \fmlInCtx{\vw}{\ophi\land \ozeta \land\bigwedge_{i\in \vec{y}} \ox_{f(i)} = \oz_{g(i)}} \ar[r,"\iota_2"]\ar[d,"\iota_1"]  & \zzeta\ar[d,"g"]\\
     \xphi \ar[r,"f"] & \ypsi
  \end{tikzcd}
\end{center}
We will take the abuse of the notation and denote both a map $f:\xphi\to \ypsi$ and the underlying renaming map between contexts $f:\vy\to \vx$
using the same letter.
Then the context $\vw$ is the pushout of the maps $f:\vy\to \vx$ and $g:\vy\to \vz$ in the category of finite sets.
The coprojections into $\vw$ are denoted $\iota_1:\vx\to\vw$ and $\iota_2:\vz\to \vw$, as below:

\begin{center}
\begin{tikzcd}
 \vec{y}\ar[d,"f"]\ar[r,"\sigma"] & \vec{z}\ar[d,"\iota_2"]\\
 \vec{x}\ar[r,"\iota_1"] & \vec{w}
\end{tikzcd}
\end{center}

That is, $\vw$ is the quotient from the disjoint union $\vx + \vz$ by gluing images in $\vx$ and $\vz$ of the same $y_i\in \vy$ together.


Then we prove the family of all the maps of the form $\iota_1$ obtained from each element $g_k$ from the covering family on $\xphi$.
The hardest part is to prove that this family satisfies the condition as required in \ref{Def:CoveringFamily}, i.e.

\begin{equation}
\phi \vdash_{\vx} \bigvee_k (\exists\vw_k.\;(\ophi\land \ozeta_k \land\bigwedge_{i\in \vec{y}} \ox_{f(i)} = \oz_{g(i)}) \land \bigwedge_{j\in\vec{x}}w_{{\iota^k_1(j)}} = x_j )
\end{equation}

from the assumption that $\{g_k\}$ is a covering family, i.e.

\[\psi\vdash_{\vy} \bigvee_k \exists \vz_k.\; \tzeta_k\land y_i = z_{g_k(i)}\]


Our strategy is to establish $\vw$ as an \emph{effective quotient} of $\vx+ \vz$.
Here we can take quotient by any relation $R$, instead of only equivalence relations.
The exactness condition will refer to the equivalence closure of the relation $R$, defined as:

\begin{definition}[Equivalence Closure]
  The equivalence closure of an equivalence relation is defined inductively.
  \lean{eqv}
  \leanok
\end{definition}

We then define effective quotient as:

\begin{definition}[Effective Quotient]
  \lean{effectiveQuotient}
  \leanok
  An effective quotient on a type $\alpha$ is a relation $R$ on $\alpha$ and a carrier set $Q$ represents of the quotient
  set by $R$. With a quotient map $q:\alpha\to Q$ and a section $s:Q\to \alpha$, such that $q\circ s = \mathbf{1}_Q$. Such that:
  \begin{itemize}
    \item soundness: if $R(a_1,a_2)$, then $q(a_1) = q(a_2)$ in the quotient set $Q$.
    \item exactness: if $q(a_1) = q(a_2)$, then $a_1$ and $a_2$ are related by the equivalence closure of $R$.
  \end{itemize}

\end{definition}

In our case, the relation $R$ will be



\begin{definition}[Relation defined by the equalities in $f(i) = g(i)$]
  Relation $R$ is the non-equivalence relation, that only relates pairs of elements in $\vx+\vz$
  iff they are images of the same element in $\vy$. That is
  \lean{WrigleyTopology.Stability.Rrel}
  \begin{equation}
    R(l_1,l_2):= \text{exists}\; y_i \in \vy \;\text{such that}\; l_1 = \inl(f(i))\; \text{and}\;  l_2 = \inl(g(i))
  \end{equation}
\end{definition}

The quotient map $\iotas: \vx+ \vz\to \vw$ will be induced by the
universal property of the coproduct, and the $s$ will be any section of it. For the moment, we assume such a section exists, although we do not define it
computably.

Let $s:\vw\to \vx+\vz$ be a section of $\iotas$, we will be able to use this $s$ to provide a witness, i.e. a substitution for the $\exists \vw_k$, to prove the
sequent that establishes the pullback family $f^*\{g_k\}$ as a covering family on $\vx$.


Assuming we have an effective quotient structure $s:\vw\to \vx+\vz$ and $\iotas:\vx+\vz\to \vw$, we take four steps to prove critical lemmas:

  \begin{equation*}
    \bigwedge_{i\in \vec{y}} x_{f(i)} = z_{\sigma(i)} \vdash
     (\bigwedge_{j\in\vec{x}}x_j = w_{\iota_1(j)}) [w_l:=(\vxz)_{s(l)}]
   \end{equation*}

and symmetrically:

\begin{equation*}
  \bigwedge_{i\in \vec{y}} x_{f(i)} = (z_{\sigma(i)} \vdash \bigwedge_{j'\in\vec{z}}z_{j'}= w_{\iota_2(j')}) [w_l:=(\vxz)_{s(l)}]
\end{equation*}

We denote the LHS of the above sequents by $E$.

For proving the above, to facilitate the use of effective quotient structure, we define a relation $S$ on the set $\vx+\vz$.
\begin{definition}[Relation defined by mapping to the same element in the quotient]
  Relation $S$ is defined by having the same image under the quotient map $\iotas:\vx+\vz\to \vw$. That is
  \lean{WrigleyTopology.Stability.Srel}
  \begin{equation}
    S(l_1,l_2):= \iotas(l_1) = \iotas(l_2)
  \end{equation}
\end{definition}



\begin{lemma}[1/4 step for covering witness lemma]\label{Lem:CovWit1}
  For each $l\in \vxz$, we have $S((\vxz)_l, (\vxz)_{s(\iotas l)})$
  \lean{WrigleyTopology.Stability.sιsSrel}
\end{lemma}

\begin{lemma}[2/4 step for covering witness lemma]\label{Lem:CovWit2}
  Prove $\iotas$ has a section $s$, hence establish $\vw$ as an effective quotient of $\vx +\vz$ via $\iotas$ and $s$.
  \lean{WrigleyTopology.Stability.ReffectiveQuotient}
  The hardest point is to provide the exactness proof.
  This is effectively proving $S\subset R^*$, where $R^*$ is the equivalence closure of $R$.
\end{lemma}


\begin{lemma}[3/4 step for covering witness lemma]\label{Lem:CovWit3}
  For all $l\in \vxz$, $r\in R^*((\vxz)_l,(\vxz)_{s(\iotas l)})$, we have
  \begin{equation*}
    E \vdash (\vxz)_l = (\vxz)_{s(\iotas l)}
  \end{equation*}
\end{lemma}

\begin{proof}
  The proof is by induction of the construction of the proof of $R^*$. (Remark: It is doable since we formalize equivalence closure
  as a type instead of a proposition.)

\begin{itemize}
    \item reflexivity should be according to proof rules for equality, same for symmetry and transitivity.
    \item base case requires us to prove
    \[E\vdash (\vxz)_{\inl(f(i))} = (\vxz)_{\inr(\sigma(i))}\]
    LHS is $x_{f(i)}$ and RHS is $z_{\sigma(i)}$, which equals by assumptions in $\Gamma$.
\end{itemize}
\end{proof}

\begin{lemma}[4/4 step for covering witness lemma]\label{Lem:CovWit4}
  \begin{equation*}
    E \vdash (\bigwedge_{j\in\vec{x}}x_j = w_{\iota_1(j)}\land \bigwedge_{j'\in\vec{z}}z_{j'}= w_{\iota_2(j')}) [w_l:=(\vxz)_{s(l)}]
  \end{equation*}
\end{lemma}
\begin{proof}
  We spell out the proof of
  \lean{WrigleyTopology.Stability.xφcoveringWitness}
  \begin{equation*}
    \bigwedge_{i\in \vec{y}} x_{f(i)} = z_{\sigma(i)} \vdash
     (\bigwedge_{j\in\vec{x}}x_j = w_{\iota_1(j)}) [w_l:=(\vxz)_{s(l)}]
   \end{equation*}
   for illustration. The other half is by symmetry.

  We take the reduction
\[(x_j = w_{\iota_1(j)}) [w_l:=(\vxz)_{s(l)}]\]
\[x_j = (\vxz)_{s(\iota_1(j))}\]
\[(\vxz)_{\inl(j)} = (\vxz)_{s(\iota_1(j))}\]
\[(\vxz)_{\inl(j)} = (\vxz)_{s(\iotas (\inl(j)))}\]
so the final equation is what we want.

This is just the conclusion of lemma \ref{Lem:CovWit3}, taking $l := \inl(j)$.

So we want to show $R^*((\vxz)_l,(\vxz)_{s(\iotas l)})$. By lemma \ref{Lem:CovWit2}, we want to show \[S((\vxz)_{\inl(j)},(\vxz)_{s(\iotas (\inl(j)))})\].

By definition of $S$, it is to prove:

\[\iotas (\inl(j)) = \iotas (s(\iotas (\inl(j)))) \]

$\iotas\circ s = 1$ cancels out. Both LHS and RHS is $\iota_1(j)$.

\end{proof}




\begin{lemma}[Witness for the existential quantification in lemma \ref{Lemma:stabCovFam}]\label{Lem:Witness}
  \lean{WrigleyTopology.Stability.coveringWitness}
  \begin{equation*}
    \phi\land \zeta\land \bigwedge_{i\in \vec{y}} x_{f(i)} = z_{\sigma(i)}\vdash_{\vx + \vz} (\overline{\phi}\land \overline{\zeta}\land\bigwedge_{j\in\vec{x}} x_j = w_{\iota_1(j)})[w_l:=(\vxz)_{s(l)}]
  \end{equation*}
\end{lemma}

\begin{proof}
  Let $\Gamma$ denote the LHS of the sequent. We prove
  \begin{equation*}
    \Gamma \vdash (\bigwedge_{j\in\vec{x}}x_j = w_{\iota_1(j)}\land \bigwedge_{j'\in\vec{z}}z_{j'}= w_{\iota_2(j')}) [w_l:=(\vxz)_{s(l)}]
  \end{equation*}

  By the lemma above, we have:

  \begin{equation*}
   \bigwedge_{i\in \vec{y}} x_{f(i)} = z_{\sigma(i)} \vdash
    (\bigwedge_{j\in\vec{x}}x_j = w_{\iota_1(j)}) [w_l:=(\vxz)_{s(l)}]
  \end{equation*}

  and

  \begin{equation*}
    \bigwedge_{i\in \vec{y}} x_{f(i)} = z_{\sigma(i)} \vdash
    (\bigwedge_{j'\in\vec{z}}z_{j'}= w_{\iota_2(j')}) [w_l:=(\vxz)_{s(l)}]
  \end{equation*}
  This will prove the equational part of the RHS of lemma \ref{Lem:Witness}, and by congruence of equalities we have:
  Then as $\ophi := \phi[w_{\iota(j)}/x_j]$, we have
  \begin{equation*}
    \phi\land ((\bigwedge_{j\in\vec{x}}x_j = w_{\iota_1(j)})[w_l:=(\vxz)_{s(l)}])  \vdash_{\vx + \vz} \ophi [w_l:=(\vxz)_{s(l)}]
  \end{equation*}


  %The proof will need some constructs about \emph{effective quotient}. Relevant definitions and lemmas are below.





\end{proof}

\begin{lemma}[Stability of covering families]\label{Lemma:stabCovFam}
  If a family $\{g_k\}\to \vec{y}\mid \psi$ is a covering family,
  then for each $f: \vec{x}\mid \phi\to \vec{y}\mid\psi$, the pullback of this covering is again a covering family.
\end{lemma}
\begin{proof}
  We want to show that from:
  \[\psi\vdash_{\vy} \bigvee_k \exists \vz_k.\; \tzeta_k\land y_i = z_{g_k(i)}\]
  that
  \[\phi \vdash_{\vx} \bigvee_k (\exists\vw_k.\;(\ophi\land \ozeta_k \land\bigwedge_{i\in \vec{y}} \ox_{f(i)} = \oz_{g(i)}) \land \bigwedge_{j\in\vec{x}}w_{{\iota^k_1(j)}} = x_j )\]

  We only spell out how to reduce the goal to lemma \ref{Lem:Witness}. After this reduction, we can drop the index $k$ in the proof.

  By definition of map between formulas in context, $f:\xphi\to \ypsi$ gives

  \begin{equation}\label{e1}
   \phi\vdash_{\vx} \psi[x_{f(i)}/y_i]
  \end{equation}


  As $\{g_k:\zzeta\to \ypsi\}$ is a covering family on $\ypsi$, by definition \ref{Def:CoveringFamily}, we have
  \[\psi\vdash_{\vy} \bigvee_k \exists z_k.\;\zeta_k \land \bigwedge_{i\in \vy} y_i = z_{g_k(i)}\]

  By the proof rule on substitution, we have


  \begin{equation}\label{e2}
  \psi [x_{f(i)}/y_i]\vdash_{\vx} (\bigvee_k \exists z_k.\;\zeta_k \land \bigwedge_{i\in \vy} y_i = z_{g_k(i)})  [x_{f(i)}/y_i]
  \end{equation}


  where the RHS reduces to, by definition of substitution
  \[\bigvee_k \exists z_k.\;\zeta_k \land \bigwedge_{i\in \vy} x_{f(i)} = z_{g_k(i)}\]

  Applying the cut rule on \ref{e1} and \ref{e2} gives
  \[\phi\vdash_{\vx} \bigvee_k \exists z_k.\;\zeta_k \land \bigwedge_{i\in \vy} x_{f(i)} = z_{g_k(i)}\]

  It is sufficient to show
  \begin{equation*}
    \phi\land \bigvee_k \exists z_k.\;\zeta_k \land \bigwedge_{i\in \vy} x_{f(i)} = z_{g_k(i)}\vdash_{\vx}
      \bigvee_k \exists \vw_k. \;(\ophi\land \ozeta_k \land\bigwedge_{i\in \vec{y}} \ox_{f(i)} = \oz_{g(i)}) \land \bigwedge_{j\in\vec{x}}w_{{\iota^k_1(j)}} = x_j
  \end{equation*}

  Frobinus and distributivity transform the LHS into
  \begin{equation}
     \bigvee_k \exists z_k. \phi\land \zeta_k \land \bigwedge_{i\in \vy} x_{f(i)} = z_{g_k(i)}
  \end{equation}


  Elimination of existential reduces it into:
  \begin{equation*}
    \phi\land \zeta_k \land \bigwedge_{i\in \vy} x_{f(i)} = z_{g_k(i)} \vdash_{\vx + \vz}
    \exists \vw_k.\;(\ophi\land \ozeta_k \land\bigwedge_{i\in \vec{y}} \ox_{f(i)} = \oz_{g(i)}) \land \bigwedge_{j\in\vec{x}}w_{{\iota^k_1(j)}} = x_j
  \end{equation*}

  The witness is given by any section of the quotient map $[\iota_1,\iota_2]:\vx + \vz \to \vw$. It reduces the goal to the lemma \ref{Lem:Witness}.

\end{proof}

\begin{theorem}[Stability of covering sieves]
  If a family $\{g_K\}\to \vec{y}\mid \psi$ is a covering sieve, then the pullback sieve $f^*\{g_K\}$ is a covering sieve on $\xphi$.
\end{theorem}

\begin{proof}
  If $\{g_K\}\to \vec{y}\mid \psi$ is a covering sieve, then it contains a covering family $\{g_k: \to \vec{y}\mid \psi\}$.
  %
  For each of the map $g_{k}:\vec{z_k}\mid \zeta_k$ in the covering family, we consider the object
  \[C:=\vec{w}\mid \ophi\land \ozeta \land\bigwedge_{i\in \vec{y}} \ox_{f(i)} = \oz_{g(i)}\].

  Here we have $\vec{w}:= \vec{x}+ \vec{z}$ is the pushout of $f$ and $\sigma$.

  \[
   \begin{tikzcd}
     \vec{w}\mid \phi\land \zeta \ar[r,"\iota_2"]\ar[d,"\iota_1"]  & \vec{z}\mid \zeta\ar[d,"g"]\\
     \vec{x}\mid \phi \ar[r,"f"] & \vec{y}\mid \psi
   \end{tikzcd}
  \]

  The underlying map $\iota_1:\vx\to\vw$ actually defines a map $C \to \xphi$.
  %
  For each of $g_k$, we have such a map. And such maps form a covering family on $\xphi$.
  %
  We will prove that we have commutative squares:

  \[
     \begin{tikzcd}
        \vec{w}\mid \ophi\land \ozeta \land\bigwedge_{i\in \vec{y}} \ox_{f(i)} = \oz_{g(i)} \ar[r,"\iota_2"]\ar[d,"\iota_1"]  & \vec{z}\mid \zeta\ar[d,"g"]\\
        \vec{x}\mid \phi \ar[r,"f"] & \vec{y}\mid \psi
     \end{tikzcd}
  \]
  Observe from the above square that the composition $f\circ \iota_1$ is equal to a pre-composition of arrow with an arrow $g\in\{g_K\}$, so each of the composition
   $f\circ \iota_1 \in \{g_K\}$.
  %
  This proves the maps of the form $\iota_1$ are all in the pullback sieve $f^*\{g_K\}$.
  %
  Hence it is sufficient to prove these $\iota_1$'s form a covering family of $\xphi$.
  %
  We will prove it as the lemma \ref{Lemma:stabCovFam} below.
\end{proof}


\begin{comment}
The formulation of stability is centered around the following commutative square.

\[
\begin{tikzcd}
   \vec{w}\mid \phi\land \zeta \ar[r,"\iota_2"]\ar[d,"\iota_1"]  & \vec{z}\mid \zeta\ar[d,"\sigma"]\\
   \vec{x}\mid \phi \ar[r,"f"] & \vec{y}\mid \psi
\end{tikzcd}
\]

where $\vec{w}:= \vec{x}+ \vec{z}$ is the pushout of $f$ and $\sigma$.

with the abuse of notation of denoting a map between formulas and the underlying map between contexts using the same letter, $\vec{w}$ is constructed as:
\[
\begin{tikzcd}
 \vec{y}\ar[d,"f"]\ar[r,"\sigma"] & \vec{z}\ar[d,"\iota_2"]\\
 \vec{x}\ar[r,"\iota_1"] & \vec{w}
\end{tikzcd}
\]






To prove the pullback of a covering sieve is again a covering sieve, we consider a covering family $\{\rho_k\}\to \vec{y}\mid \psi$, and prove that the pullback of this cover



%added


Define relations $R$ and $S$ as follows:

For each $u_1,u_2\in \vec{x}+\vec{z}$,
\[S(u_1,u_2):= [\iota_1,\iota_2]u_1 =  [\iota_1,\iota_2]u_2\]

i.e. $S$ relates $u_1$ and $u_2$ iff $u_1$ is identified with $u_2$ in the quotient.

\[R(inl(x_{f(i)}),inr(z_{\sigma(i)}))\]

Goal: Proving the sequent
\[\phi\land \zeta\land \bigwedge_{i\in \vec{y}} x_{f(i)} = z_{\sigma(i)}\vdash (\overline{\phi}\land \overline{\zeta}\land\bigwedge_{j\in\vec{x}} x_j = w_{\iota_1(j)})[w_l:=(\vxz)_{s(l)}] \]

Let $\Gamma :=$ LHS of the above sequent.

This is achieved in four steps.

Lemma 1 : For each $l\in \vxz$, we have $S((\vxz)_l, (\vxz)_{s(\iotas l)})$

Lemma 2 : Provide the proof of \emph{exact} in the structure \emph{effectiveQuotient}. This is effectively proving $S\subset R^*$, where $R^*$ is the equivalence closure of $R$.

Here $\alpha:= \vxz$, $\quot:= \iotas$

Lemma 3 : Prove for all $l\in \vxz$, $r\in R^*((\vxz)_l,(\vxz)_{s(\iotas l)})$, we have


\[\Gamma \vdash (\vxz)_l = (\vxz)_{s(\iotas l)}\].

The proof is by induction of the construction of the proof of $R^*$.

\begin{itemize}
    \item reflexivity should be according to proof rules for equality, same for symmetry and transitivity.
    \item base case requires us to prove
    \[\Gamma\vdash (\vxz)_{inl(f(i))} = (\vxz)_{inr(\sigma(i))}\]
    LHS is $x_{f(i)}$ and RHS is $z_{\sigma(i)}$, which equals by assumptions in $\Gamma$.
\end{itemize}

Lemma 4 : Prove
\[\Gamma \vdash (\bigwedge_{j\in\vec{x}}x_j = w_{\iota_1(j)}\land \bigwedge_{j'\in\vec{z}}z_{j'}= w_{\iota_2(j')}) [w_l:=(\vxz)_{s(l)}] \]
from Lemma 3.

We take the reduction
\[(x_j = w_{\iota_1(j)}) [w_l:=(\vxz)_{s(l)}]\]
\[x_j = (\vxz)_{s(\iota_1(j))}\]
\[(\vxz)_{inl(j)} = (\vxz)_{s(\iota_1(j))}\]
\[(\vxz)_{inl(j)} = (\vxz)_{s(\iotas (inl(j)))}\]
so the final equation is what we want.

This is just the conclusion of Lemma 3, taking $l := inl(j)$.

So we want to show $R^*((\vxz)_l,(\vxz)_{s(\iotas l)})$. By Lemma 2, we want to show \[S((\vxz)_{inl(j)},(\vxz)_{s(\iotas (inl(j)))})\].

By definition of $S$, it is to prove:

\[\iotas (inl(j)) = \iotas (s(\iotas (inl(j)))) \]

$\iotas\circ s = 1$ cancels out. Both LHS and RHS is $\iota_1(j)$.
\end{comment}
%added