% In this file you should put the actual content of the blueprint.
% It will be used both by the web and the print version.
% It should *not* include the \begin{document}
%
% If you want to split the blueprint content into several files then
% the current file can be a simple sequence of \input. Otherwise It
% can start with a \section or \chapter for instance.


\chapter{Big Picture}

\chapter{Monosorted Geometric Logic}
\section{Geometric Signature, Theories and Logic}

\begin{definition}
  %Lean code should be a datatype, how to link that?
A geometric signature contains the information of 
\end{definition}

%\section{Syntactic Site for a Geometric Theory} Previously @KM: is that okay?

\newcommand{\thT}{\ensuremath{\mathcal{T}}}
\newcommand{\fmlInCtx}[2]{\{ #1 | #2 \}}
\newcommand{\subs}[2]{{#1}[{#1}]}
\newcommand{\Univ}{\ensuremath{\mathscr{U}}}
%added
\newcommand{\quot}{\text{quot}}
\newcommand{\vxz}{\vec{x}+\vec{z}}
\newcommand{\vw}{\vec{w}}
\newcommand{\vx}{\vec{x}}
\newcommand{\vy}{\vec{y}}
\newcommand{\iotas}{\ensuremath{[\iota_1,\iota_2]}}
\newcommand{\xphi}{\vec{x}\mid\phi}
\newcommand{\ypsi}{\vec{y}\mid\psi}
\newcommand{\zzeta}{\vec{z}\mid\zeta}
\newcommand{\ophi}{\overline{\phi}}
\newcommand{\ozeta}{\overline{\zeta}}
\newcommand{\ox}{\overline{x}}
\newcommand{\oz}{\overline{z}}
%added

For the whole section, we fix a geometric theory \thT{} over some universe \Univ{}.

\section{Syntactic Category}
\begin{definition}[Syntactic Category]
  \lean{Joshua.fmlInCtx, Joshua.fmlMap, Joshua.categoryStruct}
  \leanok
  The syntactic category associated to $\mathcal{T}$ has:
  \begin{itemize}
    \item as objects, pairs $\fmlInCtx{\vec{x}}{\varphi}$ of a context $\vec{x} \coloneq x_1, \ldots, x_{n}$ of variables and a geometric formula $\vec{x} \vdash_{\thT} \varphi$ over \thT{};
    \item as morphisms between $\fmlInCtx{\vec{x}}{\varphi}$ and $\fmlInCtx{\vec{y}}{\psi}$, a renaming $\rho : \vec{y} \to \vec{x}$ such that $\vec{x} | \varphi \vdash \subs{\psi}{\rho}$.
  \end{itemize}
\end{definition}



\section{Syntactic Site}
\begin{definition}[Covering family]
  \label{Def:CoveringFamily}
  \lean{WrigleyTopology.CoveringFamily}
  \leanok
  A family of maps $(\rho_k : \fmlInCtx{\vec{x}_k}{\varphi_k}\to \fmlInCtx{\vec{y}}{\psi})_{k \in K}$ with $K \in \Univ$ is covering when
  $\vec{y} | \psi \vdash \bigvee_{k \in K} (\exists \vec{x}_k, \phi \wedge \bigwedge_{i \in \vec{y}} y_i = x_{\rho_k(i)}$).
\end{definition}

This covering family defines a syntactic site:

\begin{definition}[Syntactic Site]
  \lean{WrigleyTopology.SyntacticTopology}
  Define a sieve to be a covering sieve when it contains a covering family as defined in Definition \ref{Def:CoveringFamily}. This requires us to prove the following three lemmas.
\end{definition}

We prove these three conditions in the following three subsections.
\subsection{Identity}
\begin{lemma}[Identity map its own consists of a covering family]
  \lean{WrigleyTopology.id_covers}
  For each object $\vec{x}\mid \phi$ in the syntactic category, the identity map $\{\mathbf{1}_{\vec{x}\mid\phi}\}$ is a covering family.
\end{lemma}

\subsection{Transitivity}
\begin{lemma}[Transitivity]
  \lean{WrigleyTopology.Transitivity.transitivity}
  Given a formula in context $\xphi$ and a two sieves $S,R$ on it, where $S$ is a covering sieve, then if for each arrow $f:\ypsi\to\xphi$ in the
  covering sieve $S$, the pullback of $R$ along $f$ is a covering sieve on $\ypsi$, then $R$ itself is a covering sieve.
\end{lemma}



\subsection{Stability}



\begin{theorem}[Stability of covering sieves]
  If a family $\{g_K\}\to \vec{y}\mid \psi$ is a covering sieve, then the pullback sieve $f^*\{g_k\}$ is a covering sieve on $\xphi$.
\end{theorem}

\begin{proof}
  If $\{g_K\}\to \vec{y}\mid \psi$ is a covering sieve, then it contains a covering family $\{g_k: \to \vec{y}\mid \psi\}$.
  %
  For each of the map $g_{k}:\vec{z_k}\mid \zeta_k$ in the covering family, we consider the object 
  \[\vec{w}\mid \ophi\land \ozeta \land\bigwedge_{i\in \vec{x}} \ox_{f(i)} = \oz_{g(i)}\]. 

  Here we have $\vec{w}:= \vec{x}+ \vec{z}$ is the pushout of $f$ and $\sigma$.
\end{proof}

\begin{lemma}[Stability of covering families]
  If a family $\{\rho_k\}\to \vec{y}\mid \psi$ is a covering family, then for each $f: \vec{x}\mid \phi\to \vec{y}\mid\psi$, the pullback of this covering is again a covering family.
\end{lemma}





The formulation of stability is centered around the following commutative square.

\[
\begin{tikzcd}
   \vec{w}\mid \phi\land \zeta \ar[r,"\iota_2"]\ar[d,"\iota_1"]  & \vec{z}\mid \zeta\ar[d,"\sigma"]\\
   \vec{x}\mid \phi \ar[r,"f"] & \vec{y}\mid \psi
\end{tikzcd}
\]

where $\vec{w}:= \vec{x}+ \vec{z}$ is the pushout of $f$ and $\sigma$.

with the abuse of notation of denoting a map between formulas and the underlying map between contexts using the same letter, $\vec{w}$ is constructed as:
\[
\begin{tikzcd}
 \vec{y}\ar[d,"f"]\ar[r,"\sigma"] & \vec{z}\ar[d,"\iota_2"]\\
 \vec{x}\ar[r,"\iota_1"] & \vec{w}
\end{tikzcd}
\]






To prove the pullback of a covering sieve is again a covering sieve, we consider a covering family $\{\rho_k\}\to \vec{y}\mid \psi$, and prove that the pullback of this cover

\begin{definition}[Definition of the relation $R$ as in the informal proof]
\end{definition}

%added


Define relations $R$ and $S$ as follows:

For each $u_1,u_2\in \vec{x}+\vec{z}$,
\[S(u_1,u_2):= [\iota_1,\iota_2]u_1 =  [\iota_1,\iota_2]u_2\]

i.e. $S$ relates $u_1$ and $u_2$ iff $u_1$ is identified with $u_2$ in the quotient.

\[R(inl(x_{f(i)}),inr(z_{\sigma(i)}))\]

Goal: Proving the sequent
\[\phi\land \zeta\land \bigwedge_{i\in \vec{y}} x_{f(i)} = z_{\sigma(i)}\vdash (\overline{\phi}\land \overline{\zeta}\land\bigwedge_{j\in\vec{x}} x_j = w_{\iota_i(j)})[w_l:=(\vxz)_{s(l)}] \]

Let $\Gamma :=$ LHS of the above sequent.

This is achieved in four steps.

Lemma 1 : For each $l\in \vxz$, we have $S((\vxz)_l, (\vxz)_{s(\iotas l)})$

Lemma 2 : Provide the proof of \emph{exact} in the structure \emph{effectiveQuotient}. This is effectively proving $S\subset R^*$, where $R^*$ is the equivalence closure of $R$.

Here $\alpha:= \vxz$, $\quot:= \iotas$

Lemma 3 : Prove for all $l\in \vxz$, $r\in R^*((\vxz)_l,(\vxz)_{s(\iotas l)})$, we have


\[\Gamma \vdash (\vxz)_l = (\vxz)_{s(\iotas l)}\].

The proof is by induction of the construction of the proof of $R^*$.

\begin{itemize}
    \item reflexivity should be according to proof rules for equality, same for symmetry and transitivity.
    \item base case requires us to prove
    \[\Gamma\vdash (\vxz)_{inl(f(i))} = (\vxz)_{inr(\sigma(i))}\]
    LHS is $x_{f(i)}$ and RHS is $z_{\sigma(i)}$, which equals by assumptions in $\Gamma$.
\end{itemize}

Lemma 4 : Prove
\[\Gamma \vdash (\bigwedge_{j\in\vec{x}}x_j = w_{\iota_1(j)}\land \bigwedge_{j'\in\vec{z}}z_{j'}= w_{\iota_2(j')}) [w_l:=(\vxz)_{s(l)}] \]
from Lemma 3.

We take the reduction
\[(x_j = w_{\iota_1(j)}) [w_l:=(\vxz)_{s(l)}]\]
\[x_j = (\vxz)_{s(\iota_1(j))}\]
\[(\vxz)_{inl(j)} = (\vxz)_{s(\iota_1(j))}\]
\[(\vxz)_{inl(j)} = (\vxz)_{s(\iotas (inl(j)))}\]
so the final equation is what we want.

This is just the conclusion of Lemma 3, taking $l := inl(j)$.

So we want to show $R^*((\vxz)_l,(\vxz)_{s(\iotas l)})$. By Lemma 2, we want to show \[S((\vxz)_{inl(j)},(\vxz)_{s(\iotas (inl(j)))})\].

By definition of $S$, it is to prove:

\[\iotas (inl(j)) = \iotas (s(\iotas (inl(j)))) \]

$\iotas\circ s = 1$ cancels out. Both LHS and RHS is $\iota_1(j)$.

%added