% In this file you should put the actual content of the blueprint.
% It will be used both by the web and the print version.
% It should *not* include the \begin{document}
%
% If you want to split the blueprint content into several files then
% the current file can be a simple sequence of \input. Otherwise It
% can start with a \section or \chapter for instance.

\section{Geometric Signature, Theories and Logic}



\section{Syntactic Site for a Geometric Theory}

\newcommand{\thT}{\ensuremath{\mathcal{T}}}
\newcommand{\fmlInCtx}[2]{\{ #1 | #2 \}}
\newcommand{\subs}[2]{{#1}[{#1}]}
\newcommand{\Univ}{\ensuremath{\mathscr{U}}}
%added
\newcommand{\quot}{\text{quot}}
\newcommand{\vxz}{\vec{x}+\vec{z}}
\newcommand{\iotas}{\ensuremath{[\iota_1,\iota_2]}}
%added

For the whole section, we fix a geometric theory \thT{} over some universe \Univ{}.


\begin{definition}[Syntactic Category]
  \lean{Joshua.fmlInCtx, Joshua.fmlMap, Joshua.categoryStruct}
  \leanok
  The syntactic category associated to $\mathcal{T}$ has:
  \begin{itemize}
    \item as objects, pairs $\fmlInCtx{\vec{x}}{\varphi}$ of a context $\vec{x} \coloneq x_1, \ldots, x_{n}$ of variables and a geometric formula $\vec{x} \vdash_{\thT} \varphi$ over \thT{};
    \item as morphisms between $\fmlInCtx{\vec{x}}{\varphi}$ and $\fmlInCtx{\vec{y}}{\psi}$, a renaming $\rho : \vec{y} \to \vec{x}$ such that $\vec{x} | \varphi \vdash \subs{\psi}{\rho}$.
  \end{itemize}
\end{definition}

\begin{definition}[Covering family]
  \lean{Josua.CoveringFamily, Joshua.cover_from_over}
  \leanok
  A family of maps $(\rho_k : \fmlInCtx{\vec{x}_k}{\varphi_k}\to \fmlInCtx{\vec{y}}{\psi})_{k \in K}$ with $K \in \Univ$ is covering when
  $\vec{y} | \psi \vdash \bigvee_{k \in K} \exists \vec{x}_k, \phi \wedge \bigwedge_{i \in \vec{y}} y_i = x_{\rho_k(i)}$.
\end{definition}

\begin{lemma}[Stability of covering families]
\end{lemma}

%added


Define relations $R$ and $S$ as follows:

For each $u_1,u_2\in \vec{x}+\vec{z}$, 
\[S(u_1,u_2):= [\iota_1,\iota_2]u_1 =  [\iota_1,\iota_2]u_2\] 

i.e. $S$ relates $u_1$ and $u_2$ iff $u_1$ is identified with $u_2$ in the quotient.

\[R(inl(x_{f(i)}),inr(z_{\sigma(i)}))\]

Goal: Proving the sequent 
\[\phi\land \zeta\land \bigwedge_{i\in \vec{y}} x_{f(i)} = z_{\sigma(i)}\vdash (\overline{\phi}\land \overline{\zeta}\land\bigwedge_{j\in\vec{x}} x_j = w_{\iota_i(j)})[w_l:=(\vxz)_{s(l)}] \]

Let $\Gamma :=$ LHS of the above sequent.

This is achieved in four steps.

Lemma 1 : For each $l\in \vxz$, we have $S((\vxz)_l, (\vxz)_{s(\iotas l)})$

Lemma 2 : Provide the proof of \emph{exact} in the structure \emph{effectiveQuotient}. This is effectively proving $S\subset R^*$, where $R^*$ is the equivalence closure of $R$.

Here $\alpha:= \vxz$, $\quot:= \iotas$

Lemma 3 : Prove for all $l\in \vxz$, $r\in R^*((\vxz)_l,(\vxz)_{s(\iotas l)})$, we have 


\[\Gamma \vdash (\vxz)_l = (\vxz)_{s(\iotas l)}\]. 

The proof is by induction of the construction of the proof of $R^*$.

\begin{itemize}
    \item reflexivity should be according to proof rules for equality, same for symmetry and transitivity.
    \item base case requires us to prove
    \[\Gamma\vdash (\vxz)_{inl(f(i))} = (\vxz)_{inr(\sigma(i))}\]
    LHS is $x_{f(i)}$ and RHS is $z_{\sigma(i)}$, which equals by assumptions in $\Gamma$.
\end{itemize}

Lemma 4 : Prove 

\[\Gamma \vdash (\bigwedge_{j\in\vec{x}}x_j = w_{\iota_1(j)}\land \bigwedge_{j'\in\vec{z}}z_{j'}= w_{\iota_2(j')}) [w_l:=(\vxz)_{s(l)}] \]


%added